\chapter[Objetivos]{Objetivos}

  \emph{INSERIR AQUI UMA INTRODUÇÃO AO OBJETIVO}

\section{Objetivo Geral}\label{sec-objetivoGeral}

Desenvolver métodos para obtenção, limpeza e apresentação de dados gerados por 
um dispositivo de prototipagem, utilizando a linguagem de programaçao R e o 
\emph{hardware open source} Arduíno.

\section{Objetivo Específico}\label{sec-objetivoEspecifico}

\begin{alineas}
  
  \item Identificar pacotes do R que permitam interação com interfaces de hardware.
  \item Descrever as formas de comunicação e obtenção de dados entre o R e o hardware.
  \item Sugerir melhorias nos pacotes de comunicação serial.
  \item Aplicar conceitos de limpeza de dados aos dados lidos.
  \item Avaliar técnicas existentes de organização e limpeza de dados.
  \item Identificar pacotes do R que permitam apresentar dados de forma dinâmica e 
  possivelmente em tempo real.
  \item Aplicar métodos de apresentação e resumo de dados.
  \item Implementar servidor Web para visualização dos dados em tempo real.
  
\end{alineas}