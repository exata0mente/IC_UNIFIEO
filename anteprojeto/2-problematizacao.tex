\chapter*[Problema de Pesquisa]{Problema de Pesquisa}
\addcontentsline{toc}{chapter}{Problema de Pesquisa}

Como integrar leitura, tratamento e apresentação de dados de um Arduíno 
utilizando a linguagem R? Com base em minhas pesquisas identifiquei 
pouquíssimos conteúdos referente a leitura de dados de um Arduíno tendo o R como 
interface. Das postagens que encontrei, a concepção da leitura dos dados 
necessitava de [diversas 
interfaces]
(https://magesblog.com/post/2012-10-02-connecting-real-world-to-r-with-arduino/) 
como outras linguagens e softwares, tirando a atratividade para pessoas que são 
novatas no assunto (conhecem o R mas conhecem pouco de Arduino, ou conhecem o 
Arduíno mas estão aprendendo R).

**Por quê utilizar o R?** A linguagem de programação R, um dialeto da linguagem 
S, é amplamente utilizada em análise de dados e modelagem estatítica. Possui 
código aberto sob licença GNU/GPL, resultando em milhares de pacotes criados 
pela comunidade, que atendem diversas áreas específicas. Seu uso era voltado 
mais para a área de pesquisa mas acabou se difundido para a indústria. Hoje há 
diversas mídias especializadas que mostram que a linguagem R cresce muito e de 
forma constante. Um exemplo é que, de acordo com o [ranking de 2017 do 
IEEE]
(https://spectrum.ieee.org/computing/software/the-2017-top-programming-languages
) a linguagem R foi considerada a 6ª melhor linguagem de programação do ano 
![Figura 1](r_ranking_I3E.jpeg), superando linguagens populares como PHP e 
JavaScript.
Na tópico Ciência de Dados, R e Python travam um árduo duelo. É difícil definir 
qual linguagem é melhor neste ponto, havendo centenas de posts relacionados, 
alguns tendenciosos (cada um para sua linguagem favorita) e outros acalorados 
(chegando à um Fla-Flu). Meu objetivo com a linguagem R é contribuir com a 
comunidade gerando conteúdo sobre Arduíno utilizando R, além de contribuir com 
abordagens de análise de dados para a comunidade do Arduino.

**Por quê utilizar o Arduíno**? _"When you start monitoring the environment, 
something happens: You start to understand the world around you in a new way."_ 
[Gertz2012]. O baixo custo das plataformas de prototipagem, como Arduíno, a 
grande comunidade geradora de conhecimento e a filosofia *open source* 
possibilitam a qualquer pessoa, com pouquissímo conhecimento em eletrônica, 
montar sistemas de medições de variáveis ambientais apenas seguindo exemplos 
disponíveis na internet. É possível encontrar: sistemas que medem umidade do 
solo e luminosidade do local, medem turbidez, condutividade e ph da água, ruídos 
ambientais, e diversas outras variáveis, bastando ter apenas os sensores 
específicos. É fácil manusear o Arduíno, inclusive há escolas que o utilizam 
para ensino de eletrônica, robótica e programação no ensino médio.
 
Apesar de haver [soluções 
simples]
(https://magesblog.com/post/2015-02-17-reading-arduino-data-directly-into-r/) 
para o problema proposto, cabe a esta pesquisa adicionar mais elementos 
referente ao processo de análise de dados, além de fundamentação teórica nas 
etapas, tendo a linguagem de programação R como automatizadora do processo.