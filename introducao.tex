\section*{Introdução}
\addcontentsline{toc}{section}{Introdução}

%Na comunidade R, existem poucas publicações com relação ao tema proposto. As existentes, são de caráter mais específico\footnote{\url{https://www.r-bloggers.com/displaying-spatial-sensor-data-from-arduino-with-r-on-google-maps/}} veiculada em páginas pessoais ou fóruns.

% COMECE FALANDO DE ALGO SOBRE ROBÓTICA EDUCACIONAL PARA EM SEGUIDA USAR AS CITAÇÕES PARA FALAR DE USO DE HARDWARE LIVRE COMO GERADOR DE DADOS.

Quando você começa a monitorar o ambiente, algo acontece: você começa a entender o mundo ao seu redor de uma nova maneira \cite[~p.1, tradução nossa]{Gertz2012}. 

Neste projeto, será abordado as partes que compõe a leitura, tratamento e apresentação de dados, temas mais comuns aos usuários de R. Serão tratados pontos como leitura de dados de um dispositivo de prototipagem, onde será utilizado um Arduíno, métodos de limpeza e apresentação dos dados, sendo o primeiro por algoritmos ou análise dos dados e o segundo por meio de gráficos, planilhas e programação literária\footnote{\url{http://www.literateprogramming.com/}}.

Seguindo o paradigma da ciência moderna, este projeto será todo conduzido em licença aberta, utilizaremos apenas softwares e hardwares livres, além dos dados gerados, análises e \textit{scripts} implementados estarem disponíveis em repositórios abertos como GitHub\footnote{\url{https://github.com/}} e Figshare\footnote{\url{https://figshare.com/}}.

Ao término deste projeto, pretende-se alimentar mais discussões envolvendo o R e o Arduíno, fazendo com que ambos possam ser aprendidos de forma conjunta e difundida. 

\begin{citacao}[english]
  "The modern academic scientist develops code in the open, publishes data and code open source, posts preprints of their academi \c work, still submits to traditional journals, and reviews for those journals, but may also write blog posts or use social media to critique published work in post-publication review fora."\cite{Peng2015}
\end{citacao}

%A introdução seguirá a seguinte estrutura.
%    \par Importância da utilização de hardwares livres.\cite{da2016avaliaccao},\cite{bezerra2009tecnologias}, \cite{haag2005introduzir}
%    \par Importância de criar um dispositivo gerador de dados. \cite{da2016avaliaccao}
%    \par Um pouco sobre IoT, principalmente arquitetura.\cite{sethi2017internet}, \cite{santosinternet}, \cite{maia2016interface}
%    \par O arduino (wiring) \cite{Gertz2012}.
%    \par A linguagem R.\cite{oliveira2013integraccao}, \cite{Peng2015a}
    