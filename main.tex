%% abtex2-modelo-artigo.tex, v-1.9.6 laurocesar
%% Copyright 2012-2016 by abnTeX2 group at http://www.abntex.net.br/ 
%%
%% This work may be distributed and/or modified under the
%% conditions of the LaTeX Project Public License, either version 1.3
%% of this license or (at your option) any later version.
%% The latest version of this license is in
%%   http://www.latex-project.org/lppl.txt
%% and version 1.3 or later is part of all distributions of LaTeX
%% version 2005/12/01 or later.
%%
%% This work has the LPPL maintenance status `maintained'.
%% 
%% The Current Maintainer of this work is the abnTeX2 team, led
%% by Lauro César Araujo. Further information are available on 
%% http://www.abntex.net.br/
%%
%% This work consists of the files abntex2-modelo-artigo.tex and
%% abntex2-modelo-references.bib
%%

% ------------------------------------------------------------------------
% ------------------------------------------------------------------------
% abnTeX2: Modelo de Artigo Acadêmico em conformidade com
% ABNT NBR 6022:2003: Informação e documentação - Artigo em publicação 
% periódica científica impressa - Apresentação
% ------------------------------------------------------------------------
% ------------------------------------------------------------------------

\documentclass[
	% -- opções da classe memoir --
	article,			% indica que é um artigo acadêmico
	11pt,				% tamanho da fonte
	oneside,			% para impressão apenas no recto. Oposto a twoside
	a4paper,			% tamanho do papel. 
	% -- opções da classe abntex2 --
	%chapter=TITLE,		% títulos de capítulos convertidos em letras maiúsculas
	%section=TITLE,		% títulos de seções convertidos em letras maiúsculas
	%subsection=TITLE,	% títulos de subseções convertidos em letras maiúsculas
	%subsubsection=TITLE % títulos de subsubseções convertidos em letras maiúsculas
	% -- opções do pacote babel --
	english,			% idioma adicional para hifenização
	brazil,				% o último idioma é o principal do documento
	sumario=tradicional
	]{abntex2}


% ---
% PACOTES
% ---

% ---
% Pacotes fundamentais 
% ---
\usepackage{lmodern}			% Usa a fonte Latin Modern
\usepackage[T1]{fontenc}		% Selecao de codigos de fonte.
\usepackage[utf8]{inputenc}		% Codificacao do documento (conversão automática dos acentos)
\usepackage{indentfirst}		% Indenta o primeiro parágrafo de cada seção.
\usepackage{nomencl} 			% Lista de simbolos
\usepackage{color}				% Controle das cores
\usepackage{graphicx}			% Inclusão de gráficos
\usepackage{microtype} 			% para melhorias de justificação
% ---
		
% ---
% Pacotes adicionais, usados apenas no âmbito do Modelo Canônico do abnteX2
% ---
\usepackage{lipsum}				% para geração de dummy text
% ---
		
% ---
% Pacotes de citações
% ---
\usepackage[brazilian,hyperpageref]{backref}	 % Paginas com as citações na bibl
\usepackage[alf]{abntex2cite}	% Citações padrão ABNT
% ---

% ---
% Configurações do pacote backref
% Usado sem a opção hyperpageref de backref
\renewcommand{\backrefpagesname}{Citado na(s) página(s):~}
% Texto padrão antes do número das páginas
\renewcommand{\backref}{}
% Define os textos da citação
%\renewcommand*{\backrefalt}[4]{
%	\ifcase #1 %
%		Nenhuma citação no texto.%
%	\or
%		Citado na página #2.%
%	\else
%		Citado #1 vezes nas páginas #2.%
%	\fi}%
% ---

% ---
% Informações de dados para CAPA e FOLHA DE ROSTO
% ---
\titulo{Uso do ambiente R para análise, tratamento e apresentação em tempo real de dados coletados por um Arduíno.}
\autor{Ricardo Aparecido Bezerra Elias da Silva}
\local{Brasil}
\data{2018}
% ---

% ---
% Configurações de aparência do PDF final

% alterando o aspecto da cor azul
\definecolor{blue}{RGB}{41,5,195}

% informações do PDF
\makeatletter
\hypersetup{
     	%pagebackref=true,
		pdftitle={\@title}, 
		pdfauthor={\@author},
    	pdfsubject={Modelo de artigo científico com abnTeX2},
	    pdfcreator={LaTeX with abnTeX2},
		pdfkeywords={abnt}{latex}{abntex}{abntex2}{atigo científico}, 
		colorlinks=true,       		% false: boxed links; true: colored links
    	linkcolor=blue,          	% color of internal links
    	citecolor=blue,        		% color of links to bibliography
    	filecolor=magenta,      		% color of file links
		urlcolor=blue,
		bookmarksdepth=4
}
\makeatother
% --- 

% ---
% compila o indice
% ---
\makeindex
% ---

% ---
% Altera as margens padrões
% ---
\setlrmarginsandblock{3cm}{3cm}{*}
\setulmarginsandblock{3cm}{3cm}{*}
\checkandfixthelayout
% ---

% --- 
% Espaçamentos entre linhas e parágrafos 
% --- 

% O tamanho do parágrafo é dado por:
\setlength{\parindent}{1.3cm}

% Controle do espaçamento entre um parágrafo e outro:
\setlength{\parskip}{0.2cm}  % tente também \onelineskip

% Espaçamento simples
\SingleSpacing

% ----
% Início do documento
% ----
\begin{document}

% Seleciona o idioma do documento (conforme pacotes do babel)
%\selectlanguage{english}
\selectlanguage{brazil}

% Retira espaço extra obsoleto entre as frases.
\frenchspacing 

% ----------------------------------------------------------
% ELEMENTOS PRÉ-TEXTUAIS
% ----------------------------------------------------------

%---
%
% Se desejar escrever o artigo em duas colunas, descomente a linha abaixo
% e a linha com o texto ``FIM DE ARTIGO EM DUAS COLUNAS''.
%\twocolumn[    		% INICIO DE ARTIGO EM DUAS COLUNAS
%
%---
% página de titulo
\maketitle

% resumo em português
\begin{resumoumacoluna}
 Neste trabalho será proposto métodos de obtenção, tratamento e exibição em tempo real de dados gerados por sensores ligados a um Arduíno utilizando o ambiente de modelagem estatística e linguagem de programação R como interface. As medições serão efetuadas por um protótipo de uma mini central meteorológica que possuirá sensores de temperatura, umidade, pressão atmosférica e índice UV. Dessa forma será possível utilizar o método em outros projetos que faça leitura de variáveis ambientais. No trabalho em andamento busco utilizar os conceitos abordados por Jeef Leek (2016) sobre ciência moderna em que a disponibilização dos dados, códigos fontes e análise são beneficentes à toda comunidade, e exploro a bibliografia de Robert Peng (2015, 2016) voltada à ciência dos dados tendo o R como ambiente de modelagens. Além disso também será possível utilizar o conteúdo da pesquisa no auxílio da aprendizagem de dispositivos de hardware livre (potenciais geradores de dados) com a linguagem de programação R (comum em modelagem estatística).
 
 \vspace{\onelineskip}
 
 \noindent
 \textbf{Palavras-chave}: linguagem R. arduíno. análise de dados.
\end{resumoumacoluna}

%]  			% FIM DE ARTIGO EM DUAS COLUNAS
% ---

% ----------------------------------------------------------
% ELEMENTOS TEXTUAIS
% ----------------------------------------------------------
\textual

% ----------------------------------------------------------
% Introdução
% ----------------------------------------------------------
\section*{Introdução}
\addcontentsline{toc}{section}{Introdução}

Na comunidade R, existem poucas publicações com relação ao tema proposto. As existentes, são de caráter mais específico\footnote{\url{https://www.r-bloggers.com/displaying-spatial-sensor-data-from-arduino-with-r-on-google-maps/}} veiculada em páginas pessoais ou fóruns.

Neste projeto, será abordado as partes que compõe a leitura, tratamento e apresentação de dados, temas mais comuns aos usuários de R. Serão tratados pontos como leitura de dados de um dispositivo de prototipagem, onde será utilizado um Arduíno, métodos de limpeza e apresentação dos dados, sendo o primeiro por algoritmos ou análise dos dados e o segundo por meio de gráficos, planilhas e programação literária\footnote{\url{http://www.literateprogramming.com/}}.

Seguindo o paradigma da ciência moderna, este projeto será todo conduzido em licença aberta, utilizaremos apenas softwares e hardwares livres, além dos dados gerados, análises e scripts implementados estarem disponíveis em repositórios abertos como GitHub\footnote{\url{https://github.com/}} e Figshare\footnote{\url{https://figshare.com/}}.

Ao término deste projeto, pretende-se alimentar mais discussões envolvendo o R e o Arduíno, fazendo com que ambos possam ser aprendidos de forma conjunta e difundida. 

\begin{citacao}[english]
  "The modern academic scientist develops code in the open, publishes data and code open source, posts preprints of their academi \c work, still submits to traditional journals, and reviews for those journals, but may also write blog posts or use social media to critique published work in post-publication review fora."\cite{Peng2015}
\end{citacao}

%A introdução seguirá a seguinte estrutura.
%    \par Importância da utilização de hardwares livres.\cite{da2016avaliaccao},\cite{bezerra2009tecnologias}, \cite{haag2005introduzir}
%    \par Importância de criar um dispositivo gerador de dados. \cite{da2016avaliaccao}
%    \par Um pouco sobre IoT, principalmente arquitetura.\cite{sethi2017internet}, \cite{santosinternet}, \cite{maia2016interface}
%    \par O arduino (wiring) \cite{Gertz2012}.
%    \par A linguagem R.\cite{oliveira2013integraccao}, \cite{Peng2015a}
    
% ----------------------------------------------------------
% Seção de explicações
% ----------------------------------------------------------
\section[Objetivos]{Objetivos}

%% Ver issue https://github.com/exata0mente/Iniciacao_Cientifica_FIEO/issues/11

\subsection{Objetivo Geral}

% Propor métodos para obtenção, limpeza e apresentação de dados gerados por um dispositivo de prototipagem, utilizando a linguagem de programação R e o \emph{hardware open source} Arduíno.

Nesta pesquisa o objetivo geral é descrever a utilização da linguagem R como uma plataforma para projetos que utilizem hardware livre, tornando-a uma interface de leitura e apresentação de dados fazendo da linguagem uma ferramenta tanto para projetos mais robustos de IoT quanto para aprendizagem das tecnologias citadas.

\subsection{Objetivo Específico}

\begin{alineas}
  
  %\item Identificar pacotes do R que permitam interação com interfaces de hardware.
  %\item Descrever as formas de comunicação e obtenção de dados entre o R e o hardware.
  %\item Sugerir melhorias nos pacotes de comunicação serial.
  %\item Aplicar conceitos de limpeza de dados aos dados lidos.
  %\item Avaliar técnicas existentes de organização e limpeza de dados.
  %\item Identificar pacotes do R que permitam apresentar dados de forma dinâmica e possivelmente em tempo real.
  %\item Aplicar métodos de apresentação e resumo de dados.
  %\item Implementar um servidor web para visualização dos dados em tempo real.
    \item Contextualizar a linguagem de programação R como uma aplicação ao hardware livre. 
    \item Realizar pesquisas sobre o uso de um hardware livre no intuito de encontrar um fluxo de trabalho ou modo de funcionamento que tenha amplo uso.
    \item Definir uma abordagem da linguagem R na leitura dos dados gerados por um hardware livre.
    \item Projetar uma interface de exibição, via web, com software livre, para que os dados lidos possam ser apresentados ao usuário.
    \item Implementar e aplicar a abordagem em um estudo de caso - Batimentos Cardíacos em Tempo Real.
\end{alineas}

\section[Metodologia]{Metodologia}

O projeto foi iniciado com uma pesquisa sobre os métodos disponíveis em R para 
leitura de dados de um Arduíno. 

Inicialmente verificamos o pacote 
\emph{serial} \cite{Seilmayer2017} fornece métodos para comunicação com uma porta serial emulada. Em seguida, 
utilizando um projeto pronto de Arduíno, iremos efetuar leituras de dados e 
aplicar as técnicas de limpeza de dados. 

Quanto aos pacotes que serão utilizados na manipulação e apresentação de dados, 
\emph{dplyr}\cite{Wickham2018} e \emph{ggplot2}\cite{Wickham2009} respectivamente, será efetuado uma pesquisa para 
entendimento completo das funções fornecidas e posteriormente sua aplicação no 
projeto. Este é um dos pontos chaves do projeto já que aqui definiremos ou 
criaremos funções que ficarão em execução enquanto os dados estiverem sendo 
gerados.

Após definido os métodos de leitura, limpeza e apresentação dos dados, será 
iniciado uma pesquisa sobre ferramentas de aplicação \emph{web} que permitam a 
integração da linguagem R. Em seguida iremos montar um servidor web que recebe e 
apresenta em tempo real os dados gerados pelo Arduíno. Para isso, fará parte do 
conhecimento adquirido nesta pesquisa as tecnologias que permitem a criação, 
hospedagem e modelagem de um servidor de aplicação \emph{web}.

Ao final do projeto, pretende-se apresentar um sistema que mostre em tempo real, 
informações (gráficos, tabelas, resumos) de dados gerados por um dispositivo de 
prototipagem Arduíno.

% ---
% Finaliza a parte no bookmark do PDF, para que se inicie o bookmark na raiz
% ---
\bookmarksetup{startatroot}% 
% ---

% ---
% Conclusão
% ---
\section*{Considerações finais}
%\addcontentsline{toc}{section}{Considerações finais}

A linguagem de programação R e o dispositivo de prototipagem do Arduíno possuem semelhança quanto a sua cultura: ser "livre". 
Este tipo de filosofia permite que diversos conteúdos sejam gerados referentes a várias aplicações. A linguagem R por exemplo, é uma excelente ferramenta (desde seu ambiente, aplicações e até tipagem de sua linguagem) para tratamento de dados. Isso faz com que diversos pacotes sejam criados pela comunidade. Os pacotes dplyr e ggplot2 por exemplo são ferramentas que estão mostrando-se altamente aplicável ao projeto em andamento. Também o pacote serial pois apresenta todos os métodos que foram necessários para conexão e leitura de uma porta serial emulada, característica de Arduíno.

Os próximos passos serão de comparação de desempenho de leitura de dados com outros pacotes do R, diferentes formas de obtenção de dados (wi-fi, bluetooth, webservices) e desenvolvimento de um ambiente web para apresentação dos dados em qualquer lugar possibilitando a verificação de medições em ambientes fisicamente distantes.

%\begin{citacao}[english]
 % "When you start monitoring the environment,
  %something happens: You start to understand the world around you in a %new way." \cite{Gertz2012}
%\end{citacao}

% ----------------------------------------------------------
% ELEMENTOS PÓS-TEXTUAIS
% ----------------------------------------------------------

% ]  				% FIM DE ARTIGO EM DUAS COLUNAS
% ---

% ----------------------------------------------------------
% Referências bibliográficas
% ----------------------------------------------------------
\bibliography{refs/refs}

% ----------------------------------------------------------
% Glossário
% ----------------------------------------------------------
%
% Há diversas soluções prontas para glossário em LaTeX. 
% Consulte o manual do abnTeX2 para obter sugestões.
%
%\glossary

% ----------------------------------------------------------
% Apêndices
% ----------------------------------------------------------

% ---
% Inicia os apêndices
% ---
% ---

% ----------------------------------------------------------
% Anexos
% ----------------------------------------------------------

% ---
% Inicia os anexos
% ---
%\anexos
\end{document}
