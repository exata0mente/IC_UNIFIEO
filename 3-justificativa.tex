\chapter[Justificativa]{Justificativa}


Na comunidade R, existem poucas publicações com relação ao tema proposto. As existentes, são de carater mais 
específico\footnote{\url{https://www.r-bloggers.com/displaying-spatial-sensor-data-from-arduino-with-r-o
n-google-maps/}} veículada em páginas pessoais ou fóruns. 

Neste projeto, abordaremos as partes que compõe a leitura, tratamento e 
apresentação de dados, temas mais comuns aos usuários de R. Serão tratados 
pontos como leitura de dados de um dispositivo de prototipagem (usaremos o 
Arduino), métodos de limpeza e apresentação dos dados, sendo o primeiro por 
algoritmos ou análise dos dados e o segundo por meio de gráficos, planilhas e 
programação literária\footnote{\url{https://en.wikipedia.org/wiki/Literate_programming}}.

Seguindo o paradigma da ciência moderna, este projeto será todo conduzido em 
licença aberta, utilizaremos apenas softwares e hardwares livres, além dos dados 
gerados, análises e scripts implementados estarem disponíveis em repositórios 
abertos como GitHub\footnote{\url{https://github.com/}} e Figshare\footnote{\url{https://figshare.com/}}.

Ao término deste projeto, pretendo alimentar mais discussões envolvendo o R e o 
Arduíno, fazendo com que ambos possam ser aprendidos de forma conjunta e 
difundida. 

