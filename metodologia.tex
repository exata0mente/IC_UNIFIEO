
\section[Metodologia]{Metodologia}

O projeto foi iniciado com uma pesquisa sobre os métodos disponíveis em R para 
leitura de dados de um Arduíno. 

Inicialmente verificamos o pacote 
\emph{serial} \cite{Seilmayer2017} fornece métodos para comunicação com uma porta serial emulada. Em seguida, 
utilizando um projeto pronto de Arduíno, iremos efetuar leituras de dados e 
aplicar as técnicas de limpeza de dados. 

Quanto aos pacotes que serão utilizados na manipulação e apresentação de dados, 
\emph{dplyr}\cite{Wickham2018} e \emph{ggplot2}\cite{Wickham2009} respectivamente, será efetuado uma pesquisa para 
entendimento completo das funções fornecidas e posteriormente sua aplicação no 
projeto. Este é um dos pontos chaves do projeto já que aqui definiremos ou 
criaremos funções que ficarão em execução enquanto os dados estiverem sendo 
gerados.

Após definido os métodos de leitura, limpeza e apresentação dos dados, será 
iniciado uma pesquisa sobre ferramentas de aplicação \emph{web} que permitam a 
integração da linguagem R. Em seguida iremos montar um servidor web que recebe e 
apresenta em tempo real os dados gerados pelo Arduíno. Para isso, fará parte do 
conhecimento adquirido nesta pesquisa as tecnologias que permitem a criação, 
hospedagem e modelagem de um servidor de aplicação \emph{web}.

Ao final do projeto, pretende-se apresentar um sistema que mostre em tempo real, 
informações (gráficos, tabelas, resumos) de dados gerados por um dispositivo de 
prototipagem Arduíno.
