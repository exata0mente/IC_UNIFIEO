\section[Objetivos]{Objetivos}

%% Ver issue https://github.com/exata0mente/Iniciacao_Cientifica_FIEO/issues/11

\subsection{Objetivo Geral}

% Propor métodos para obtenção, limpeza e apresentação de dados gerados por um dispositivo de prototipagem, utilizando a linguagem de programação R e o \emph{hardware open source} Arduíno.

Nesta pesquisa o objetivo geral é descrever a utilização da linguagem R como uma plataforma para projetos que utilizem hardware livre, tornando-a uma interface de leitura e apresentação de dados fazendo da linguagem uma ferramenta tanto para projetos mais robustos de IoT quanto para aprendizagem das tecnologias citadas.

\subsection{Objetivo Específico}

\begin{alineas}
  
  %\item Identificar pacotes do R que permitam interação com interfaces de hardware.
  %\item Descrever as formas de comunicação e obtenção de dados entre o R e o hardware.
  %\item Sugerir melhorias nos pacotes de comunicação serial.
  %\item Aplicar conceitos de limpeza de dados aos dados lidos.
  %\item Avaliar técnicas existentes de organização e limpeza de dados.
  %\item Identificar pacotes do R que permitam apresentar dados de forma dinâmica e possivelmente em tempo real.
  %\item Aplicar métodos de apresentação e resumo de dados.
  %\item Implementar um servidor web para visualização dos dados em tempo real.
    \item Contextualizar a linguagem de programação R como uma aplicação ao hardware livre. 
    \item Realizar pesquisas sobre o uso de um hardware livre no intuito de encontrar um fluxo de trabalho ou modo de funcionamento que tenha amplo uso.
    \item Definir uma abordagem da linguagem R na leitura dos dados gerados por um hardware livre.
    \item Projetar uma interface de exibição, via web, com software livre, para que os dados lidos possam ser apresentados ao usuário.
    \item Implementar e aplicar a abordagem em um estudo de caso - Batimentos Cardíacos em Tempo Real.
\end{alineas}