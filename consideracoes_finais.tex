\section*{Considerações finais}
%\addcontentsline{toc}{section}{Considerações finais}

A linguagem de programação R e o dispositivo de prototipagem do Arduíno possuem semelhança quanto a sua cultura: ser "livre". 
Este tipo de filosofia permite que diversos conteúdos sejam gerados referentes a várias aplicações. A linguagem R por exemplo, é uma excelente ferramenta (desde seu ambiente, aplicações e até tipagem de sua linguagem) para tratamento de dados. Isso faz com que diversos pacotes sejam criados pela comunidade. Os pacotes dplyr e ggplot2 por exemplo são ferramentas que estão mostrando-se altamente aplicável ao projeto em andamento. Também o pacote serial pois apresenta todos os métodos que foram necessários para conexão e leitura de uma porta serial emulada, característica de Arduíno.

Os próximos passos serão de comparação de desempenho de leitura de dados com outros pacotes do R, diferentes formas de obtenção de dados (wi-fi, bluetooth, webservices) e desenvolvimento de um ambiente web para apresentação dos dados em qualquer lugar possibilitando a verificação de medições em ambientes fisicamente distantes.
