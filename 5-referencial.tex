\chapter[Referencial Teórico]{Referencial Teórico}

Nosso referencial teórico sobre a linguagem de programação R e análise de dados 
será baseada nas produções acadêmicas de \citeauthoronline{Leek2016}, \citeauthoronline{Peng2015} e 
\citeauthoronline{Caffo2015}. Os autores abordam os passos de um processo de análise de dados e 
suas respectivas ferramentas. Roger Peng em seu livro \emph{R Programming for Data 
Science}\footnote{\url{https://leanpub.com/modernscientist}}  traz os conceitos básicos para a utilização do R como linguagem de 
programação, este livro pode ser considerado um pré-requisito para a leitura dos 
livros seguintes e servirá como base para questões que envolvem programação 
estruturada. Em \emph{Exploratory Data Analysis with R}\footnote{\url{https://leanpub.com/exdata}}, Peng apresenta ferramentas 
em R que são utilizadas na análise exploratória de dados. Neste livro é 
apresentado os pacotes \emph{dplyr} e \emph{ggplot2} para manipulação e apresentação de 
dados. Além disso são apresentadas técnicas de análise, organização e limpeza de 
dados como Hierarchical Clustering, K-Means Clustering e Dimension Reduction. 
Além destas citadas, nesta pesquisa exploraremos outras técnicas de limpeza de 
dados, como as utilizadas em Mineração de Dados.

\begin{citacao}[english]
  "The modern academic scientist develops code in the open, publishes data and 
code open source, posts preprints of their academic work, still submits to 
traditional journals, and reviews for those journals, but may also write blog 
posts or use social media to critique published work in post-publication review 
fora."\cite{Peng2015}
\end{citacao}
