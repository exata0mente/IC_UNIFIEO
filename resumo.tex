\begin{resumoumacoluna}
 Neste trabalho será proposto métodos de obtenção, tratamento e exibição em tempo real de dados gerados por sensores ligados a um Arduíno utilizando o ambiente de modelagem estatística e linguagem de programação R como interface. As medições serão efetuadas por um protótipo de uma mini central meteorológica que possuirá sensores de temperatura, umidade, pressão atmosférica e índice UV. Dessa forma será possível utilizar o método em outros projetos que faça leitura de variáveis ambientais. No trabalho em andamento busco utilizar os conceitos abordados por Jeef Leek (2016) sobre ciência moderna em que a disponibilização dos dados, códigos fontes e análise são beneficentes à toda comunidade, e exploro a bibliografia de Robert Peng (2015, 2016) voltada à ciência dos dados tendo o R como ambiente de modelagens. Além disso também será possível utilizar o conteúdo da pesquisa no auxílio da aprendizagem de dispositivos de hardware livre (potenciais geradores de dados) com a linguagem de programação R (comum em modelagem estatística).
 
 \vspace{\onelineskip}
 
 \noindent
 \textbf{Palavras-chave}: linguagem R. arduíno. análise de dados.
\end{resumoumacoluna}
